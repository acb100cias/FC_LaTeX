\documentclass[addpoints]{exam}
\usepackage[spanish]{babel}
\pointpoints{punto}{puntos}
\bonuspointpoints{punto extra}{puntos extra}
\totalformat{Pregunta \thequestion: \totalpoints puntos }
\chqword{Pregunta}
\chqword{Página}
\chqword{Puntos}
\chqword{Puntos extra}
\chqword{ Puntos obtenidos}
\chqword{Total}
\usepackage[utf8x]{inputenc}
\usepackage{amsthm,amsmath,amssymb}
\decimalpoint
\begin{document}
\begin{center}
    \fbox{\fbox{\parbox{10cm}{\centering Contesta cuatro preguntas, en los espacios, debajo de ellas}}}
\end{center}

\vspace{5mm}
\makebox[0.75\textwidth]{Nombre:\enspace\hrulefill}

\vspace{5mm}
\makebox[0.75\textwidth]{Materia:\enspace\hrulefill}

\vspace{5mm}
\makebox[0.75\textwidth]{Profesor:\enspace\hrulefill}

\begin{questions}
    \question[3 \half] Pruebe que si $\delta$ es una norma en un espacio vectorial $V$, entonces induce una métrica.
    \vspace{\stretch{1}}
    \question [3 \half] Pruebe que una métrica induce una topología.
    \vspace{\stretch{1}}
    \question Considera el problema de valores iniciales
    $$
    \dot{x}=rx(1-\frac{x}{K})
    $$
    con la condición inicial $x_0=K+0.2$. Contesta
    \begin{parts}
        \part[1] Encuentra las posibles soluciones estacionarias.
        \vspace{\stretch{1}}
        \part[1] ¿Cuál es la estabilidad de  tales soluciones?
        \begin{subparts}
        \subpart Da una interpretación biológica
        \vspace{\stretch{1}}
        \subpart ¿Existe algún tipo de bifurcación?
        \vspace{\stretch{1}}
        \end{subparts}
        \vspace{\stretch{1}}
        \part[1] Esboza las soluciones
        \vspace{\stretch{1}}
    \end{parts}
%\vspace{\stretch{1}}
\droptotalpoints
\bonusquestion[1] ¿Cúal de los siguientes inventó la simetría?

\begin{oneparchoices}
    \choice Emmy Noether
    \choice Sophus Lie
    \choice Felix Klein
    \choice Esta pregunta carece de sentido    
\end{oneparchoices}

\bonusquestion[1] ¿Cúales de los siguientes conceptos pertenece a la teoría delos sistemas complejos?

\begin{checkboxes}
    \choice Criticalidad
    \choice Auto-organización
    \choice Emergencia
    \choice Singularidad
    \choice Excitabilidad
\end{checkboxes}

\end{questions}

\begin{center}
    \combinedgradetable[h][questions]
\end{center}
\clearpage

\end{document}