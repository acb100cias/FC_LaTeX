\section{Minted}

Ejemplo de código en Python:

\begin{minted}[bgcolor=LightGray]{python}
import numpy as np

def incmatrix(genl1,genl2):
    m=len(genl1)
    n=len(genl2)
    M=None
    VT=np.zeros((n*m,1))

# Calcula la matriz xor

    for i in range(m-1):
        for j in range(i+1,m):
            [r,c]=np.where(M2==M1[i,j])
            for k in range(len(r)):
                VT[(i)*n+r[k]]=1
                VT[(i)*n+c[k]]=1
                VT[(j)*n+r[k]]=1
                VT[(j)*n+c[k]]=1

                if M is none:
                    M= np.copy(VT)
                else:
                    M=np.concatenate((M,VT),1)
    return M
\end{minted}


Código en Octave desde un script

\inputminted[bgcolor=LightGray]{octave}{bitxorM.m}

Código en C

\begin{minted}[bgcolor=LightGray]{c}
#include <stdio.h>
int main() {
   printf("Hello, World!"); /*printf() outputs the quoted string*/
   return 0;
}
\end{minted}

Perl:
\inputminted[bgcolor=LightGray]{perl}{bitxorM.m}

Emacs
\inputminted[]{emacs}{bitxorM.m}

VIM
\inputminted[]{vim}{bitxorM.m}

\begin{minted}[bgcolor=LightGray]{rust}
    import numpy as np

def incmatrix(genl1,genl2):
    m=len(genl1)
    n=len(genl2)
    M=None
    VT=np.zeros((n*m,1))
\end{minted}

