\section{Diagramas con tikz}
%%%%%%%%%%%%%%%%%%%%%%%%%%%%%%%%%%%%%%%%%%%%%
\tikzstyle{startstop}=[rectangle,rounded corners, minimum width=3cm,minimum height= 1cm, text centered,draw=black, fill=red]
\tikzstyle{io}=[trapezium,trapezium left angle=70,trapezium right angle=110,minimum width=3cm,minimum height= 1cm, text centered,draw=black, fill=blue]
\tikzstyle{process}=[rectangle,minimum width=3cm,minimum height= 1cm, text centered,text width= 3cm,draw=black, fill=orange]
\tikzstyle{decision}=[diamond,minimum width=3cm,minimum height= 1cm, text centered,draw=black, fill=green]
\tikzstyle{arrow}=[thick]

%%%%%%%%%%%%%%%%%%%%%%%%%%%%%%%%%%%%%%%%%%%


\begin{tikzpicture}[node distance=2cm]
    \node (start)[startstop]{\bf Inicio};
    \node (in1)[io, below of=start]{\bf Entrada};
    \node (p1)[process, below of=in1]{\bf Proceso 1 \\ Acá empieza};
    \node (d1)[decision, below of=p1]{\bf IF};
    \node (p2a)[process,below of=d1,yshift=-1.5cm]{\bf Proceso 2 A \\ Ya casi acabamos};
     \node (p2b)[process,right of=d1,xshift=3cm]{\bf Proceso 2 B\\ Este es un proceso dummy};
    \node (o1)[io, below of=p2a]{\bf Salida};
    \node (stop)[startstop,below of=o1]{\bf Alto};
    \draw[-stealth, thick](start) -- (in1);
    \draw[-stealth, thick](in1) -- (p1);
    \draw[-stealth, thick](p1) -- (d1);
    \draw[-stealth, thick](d1) -- node[anchor=east] {\bf True} (p2a);
    \draw[-stealth, thick](d1) -- node[anchor=south] {\bf False} (p2b);
    \draw[-stealth,thick](p2b)|-(p1);
    \draw[-stealth, thick](p2a) -- (o1);
    \draw[-stealth, thick](o1) -- (stop);
\end{tikzpicture}